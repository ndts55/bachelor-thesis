\chapter{Conclusion}
% Clearly state the answer to the main research question
% Summarize and reflect on the research
% Make recommendations for future work on the topic
% Show what new knowledge you have contributed
In this work we have provided a practical example of a static analysis of Paxos using \FTMPST.
% Was habe ich gemacht? Wieso habe ich das gemacht? Was hat mir das gebracht?

\section{Summary}
% summarize what we did in this thesis.
% - presented Paxos (why Paxos? because it is basic Paxos. it is quintessential Paxos), FTMPST, additional syntax+semantics for FTMPST (which specifically? introduced "function calls as prefixes" to manipulate registers)
First, we introduced Paxos and \FTMPST.
For the specification of Paxos we chose \cite{Lamport01} because in it \citeauthor{Lamport01} describes the most basic form of the Paxos algorithm.
Paxos is designed to solve the consensus problem in a network where processes or their communications may experience failures.
To model the fault-tolerance of Paxos we used \FTMPST, which are Multiparty Session Types extended with fault-tolerance.

We extended \FTMPST with function calls as prefixes to model side effects, \eg updating a register and introduced notation to abbreviate sequences of prefixes and weakly reliable branches.
Additionally, we specified a notation for defining sorts with type parameters.

% - model: sorts, global type, functions, processes, failure patterns
%   - sorts: defined data types (sorts). explicit data types for the concepts used in Paxos. used in definition of global type. Bool, Maybe, Proposal, Promise (defined above)
The sorts $\Bool$, $\Maybe{\TypeParamA}$, $\Proposal{\TypeParamA}$, and $\Promise{\TypeParamA}$ are introduced.
These represent data types used in the global type and processes.

%   - global type: defined global type for session with one proposer and a quorum of acceptors. considers more precisely what Lamport described in Lamport01 (one proposer + n acceptors). global type required for local types (required for type check).
Next, we specified the global type for a session with one proposer and a quorum.
This mirrors the phases $\POneA$, $\POneB$ and $\PTwoA$ described in \cite{Lamport01}.
Phase $\PTwoB$ is not modelled in the global type because it contains no inter-process communication.
The global type can be projected to local types, which are needed for the type check.

%   - functions: defined functions.
To construct the processes some functions are introduced.
%       - model some aspect of Paxos, specifically aspects that were assumed in Lamport01
%       - propNumber: models disjoint sets of proposal numbers for proposers
%       - promValue: models picking the proposal with the highest proposal number
%       - anyNack, promCount: model evaluation of the gathered responses from phase 1b
%       - genQ: models construction of random quorums from the set of acceptors
Some model various aspects of the algorithm, \eg picking proposal numbers from disjoint sets of numbers, picking a proposal value from a list of promises, evaluating lists of promises, and generating sets of acceptors that constitute a quorum.
%       - gt, ge: variants of > and >= where one argument is of type Maybe a
%       - nFromProp: allows access to the proposal number of a proposal
Others allowed us to simplify working with the specified structure of the sorts in the processes, \ie accessing internal values and comparing values of different sorts.
%       - pos, posRec: models the role selection of an acceptor within a quorum
One more function was introduced to correctly assign roles to processes within a session.

%   - processes: defined processes for system init, proposer, acceptor. execute the Paxos algorithm. type check requires something to type check, this is it.
With these functions we defined processes for system initialization, proposers, and acceptors.

Each process is assigned a process index.
An acceptor subprocess only accepts a proposer's session request if its process index is in that proposer's quorum.
However, the process index is not the role the acceptor subprocess has in that session.
So, we mapped the process index of each acceptor subprocess to its position within the corresponding proposer's quorum to obtain its role.
This way proposers only communicate with acceptors inside their quorum, and we avoid communication mismatches.

Including quorums in our model raised the complexity of the system initialization specifically.
Otherwise, if all proposers communicated with all acceptors the process index would equal that process' role in a session.

The processes for proposers and acceptors followed directly from the description of the Paxos algorithm and the global type.

%   - failure patterns: implemented failure patterns. model system requirements. proof for termination requires these system requirements.
The system requirements were modelled in \FTMPST using failure patterns.
We used the failure patterns to prove termination for our model.

%   - presented an example run: show how the model works. presentation of how the model works.
In order to demonstrate how the model works we examined an example run with three acceptors and two proposers.
We described a scenario where one proposer completes the algorithm and another tries to run the algorithm and fails but retries and succeeds.
Then, we applied the appropriate reduction rules to the formulae.

% - analysis: projected global types to local types, type check, proofs for validity, agreement, termination
Finally, After defining and studying an example run of our model, we analyze it with the goal of proving validity, agreement, and termination.
%   - projected global types to local types. local types required for type check. enables us to do type check.
%   - executed type check. checks that the communication structure of our process follows the communication structure in the global type. type check ensures absence of communication mismatches.
To prove these properties we ensured our model is well-typed by type checking it.
To execute the type check we projected the global type to local types first.
%   - proofs for validity, agreement, termination. satisfaciton of our goal
Using well-typedness of our model we proved validity, agreement, and termination.
This satisfies our goal of analyzing Paxos with \FTMPST.

\section{Future Work}
% - analyse how close the model is to the actual algorithm
% disparities between Paxos phases and global type:
There are two notable differences between the global type and Paxos.
% - l1b: an acceptor that never received a message from the proposer would not send a message to the proposer
First, in phase $\POneB$ of Paxos only acceptors that received a prepare request send a message to the proposer, whereas in the global type all acceptors in the session send a message to the proposer.
% - l2a: split into two messages (branching and if Accept the proposal) where in Paxos it would just be one (but that wouldn't allow us to terminate a process because we would always have to restart the algorithm)
Second, the global type splits phase $\PTwoA$ into two messages.
A weakly reliable branching to broadcast the proposers decision about whether to restart the algorithm or send an accept request and the accept request itself.
In Paxos the proposer does not inform the acceptors of its decision to restart or send an accept request.
Future research is needed to assess how close the model is to the Paxos algorithm and how it could be improved.

Another question for future work would be how our model compares to a different model of the same algorithm.
We chose to define the global type for a single proposer and its quorum.
An alternative solution might define the global type for all proposers and acceptors and have them participate in the same session.
