\chapter{Conclusion}
% TODO Write conclusion.
% Clearly state the answer to the main research question
% Summarize and reflect on the research
% Make recommendations for future work on the topic
% Show what new knowledge you have contributed
In this work we have provided a practical example of a static analysis of Paxos using \FTMPST.

% Was habe ich gemacht? Wieso habe ich das gemacht? Was hat mir das gebracht?

% summarize what we did in this thesis.
% - presented Paxos (why Paxos? because it is basic Paxos. it is quintessential Paxos), FTMPST, additional syntax+semantics for FTMPST (which specifically? introduced "function calls as prefixes" to manipulate registers)
First, we introduced Paxos and \FTMPST.
For the specification of Paxos we chose \cite{Lamport01} because in it \citeauthor{Lamport01} describes the most basic form of the Paxos algorithm.
Paxos is designed to solve the consensus problem in a network where processes or their communications may experience failures.
To model the fault-tolerance of Paxos we used \FTMPST.
Multiparty Session Types extended to include fault-tolerance.

We extended \FTMPST with function calls as prefixes to model side effects, \eg updating a register.

% - defined: sorts, global type, functions, processes, failure patterns
% - presented an example run
% - proven that processes fulfill the global type via type check
% - proofs for validity, agreement, and termination for the model

% what difficulties did we encounter when creating the model?
% - working around the concept of a quorum made the model more complex
% - without quorums (if every proposer just communicated with every acceptor) the processes (specifically system initialization) would be simpler

% what could be future work?
% - analyse how close the model is to the actual algorithm
% - comparison with other models for the same algorithm
% - extension of the model to include learners
