\section{Paxos}
\citeauthor{Lamport01} describes the Paxos algorithm in \cite{Lamport01}.
First, he outlines the problem and then the algorithm to solve that problem.

\subsection{The Problem}
Assume a collection of processes that can propose values.
A consensus algorithm ensures that a single one among the proposed values is chosen.
If no value is proposed, then no value should be chosen.
If a value has been chosen, then processes should be able to learn the chosen value.
The safety requirements for consensus are:

\begin{itemize}
    % Validity
    \item \textbf{Validity}: Only a value that has been proposed may be chosen,
    % Agreement
    \item \textbf{Agreement}: Only a single value is chosen, and
    % Termination? I'd prefer
    % "If a value has been proposed and sufficient processes remain non-faulty, then eventually some value is chosen."
    % \item A process never learns that a value has been chosen unless it actually has been.
    \item \textbf{Termination}: If a value has been proposed and sufficient processes remain non-faulty, then eventually some value is chosen (compare to \cite{Lamport05}).
\end{itemize}

% TODO "These are the properties we will prove with the model" oder sowas in der Art.

Assume that agents can communicate with one another by sending messages.
We use the customary asynchronous, non-Byzantine model, in which:

\begin{itemize}
    \item Agents operate at arbitrary speed, may fail by stopping, and may restart.
    Since all agents may fail after a value is chosen and then restart, a solution is impossible unless some information can be remembered by an agent that has failed and restarted.
    \item Messages can take arbitrarily long to be delivered, can be duplicated, and can be lost, but they are not corrupted.
\end{itemize}

We assume two classes of agents: proposers and acceptors.

A \emph{quorum} is a set of acceptors that can choose a value proposed by a proposer \cite{Lamport06}.
For Paxos a quorum is a majority of acceptors \cite{Lamport06}.

\subsection{The Algorithm}

\begin{phase}
    $(a)$ A proposer selects a proposal number $n$ and sends a \emph{prepare} request with number $n$ to a majority of acceptors.
    
    $(b)$ If an acceptor receives a \emph{prepare} request with number $n$ greater than that of any \emph{prepare} request to which it has already responded, then it responds to the request with a promise not to accept any more proposals numbered less than $n$ and with the highest-numbered proposal (if any) that it has accepted.
\end{phase}

\begin{phase}
    $(a)$ If the proposer receives a response to its \emph{prepare} requests (numbered $n$) from a majority of acceptors, then it sends an \emph{accept} request to each of those acceptors for a proposal numbered $n$ with a value $v$, where $v$ is the value of the highest-numbered proposal among the responses, or is any value if the responses reported no proposals.

    $(b)$ If an acceptor receives an \emph{accept} request for a proposal numbered $n$, it accepts the proposal unless it has already responded to a \emph{prepare} request having a number greater than $n$.
\end{phase}

A proposer can make multiple proposals, so long as it follows the algorithm for each one.
It can abandon a proposal in the middle of the protocol at any time.
It is probably a good idea to abandon a proposal if some proposer has begun trying to issue a higher-numbered one.
Therefore, if an acceptor ignores a \emph{prepare} or \emph{accept} request because it has already received a \emph{prepare} request with a higher number, then it should probably inform the proposer, who should then abandon its proposal.
This is a performance optimization that does not affect correctness.

To guarantee progress, a distinguished proposer must be selected as the only one to try issuing proposals.
If the distinguished proposer can communicate successfully with a majority of acceptors, and if it uses a proposal with number greater than any already used, then it will succeed in issuing a proposal that is accepted.
By abandoning a proposal and trying again if it learns about some request with a higher proposal number, the distinguished proposer will eventually choose a high enough proposal number.

If enough of the system is working properly, liveness can therefore be achieved by electing a single distinguished proposer.

\subsection{The Implementation}
The Paxos algorithm \cite{Lamport98} assumes a network of processes.
In its consensus algorithm, each process plays the role of proposer, acceptor.
The algorithm chooses a leader, which plays the roles of the distinguished proposer.
The Paxos consensus algorithm is precisely the one described above, where requests and responses are sent as ordinary messages.
Stable storage, preserved during failures, is used to maintain the information that the acceptor must remember.
An acceptor records its intended response in stable storage before actually sending the response.

Different proposers choose their numbers from disjoint sets of numbers, so two different proposers never issue a proposal with the same number.
Each proposer remembers (in stable storage) the highest-numbered proposal it has tried to issue, and begins phase 1 with a higher proposal number than any it has already used.
