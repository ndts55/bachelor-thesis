\section{Paxos}
% TODO Introduce Lamports text properly.
\citeauthor{Lamport01} describes the Paxos algorithm in \cite{Lamport01}.

\subsection{Choosing a Value}
The easiest way to choose a value is to have a single acceptor agent.
A proposer sends a proposal to the acceptor, who chooses the first proposed value
that it receives.
Although simple, this solution is unsatisfactory because the
failure of the acceptor makes any further progress impossible.
So, let's try another way of choosing a value.
Instead of a single acceptor, let's use multiple acceptor agents.
A proposer sends a proposed value to a set of acceptors.
An acceptor may \emph{accept} the proposed value.
The value is chosen when a large enough set of acceptors have accepted it.
How large is large enough?
To ensure that only a single value is chosen, we can let a large enough set consist of any majority of the agents.
Because any two majorities have at least one acceptor in common, this works if an acceptor can accept at most one value.
(There is an obvious generalization of a majority that has been observed in numerous papers, apparently starting with [3].)
In the absence of failure or message loss, we want a value to be chosen even if only one value is proposed by a single proposer.
This suggests the requirement:

\begin{paxostheoremenv}[$\PaxosTheorem{1}$]
    An acceptor must accept the first proposal that it receives.
\end{paxostheoremenv}

But this requirement raises a problem.
Several values could be proposed by different proposers at about the same time, leading to a situation in which every acceptor has accepted a value, but no single value is accepted by a majority of them.
Even with just two proposed values, if each is accepted by about half the acceptors, failure of a single acceptor could make it impossible to learn which of the values was chosen.
$\PaxosTheorem{1}$ and the requirement that a value is chosen only when it is accepted by a majority of acceptors imply that an acceptor must be allowed to accept more than one proposal.
We keep track of the different proposals that an acceptor may accept by assigning a (natural) number to each proposal, so a proposal consists of a proposal number and a value.
To prevent confusion, we require that different proposals have different numbers.
How this is achieved depends on the implementation, so for now we just assume it.
A value is chosen when a single proposal with that value has been accepted by a majority of the acceptors.
In that case, we say that the proposal (as well as its value) has been chosen.
We can allow multiple proposals to be chosen, but we must guarantee that all chosen proposals have the same value.
By induction on the proposal number, it suffices to guarantee:

\begin{paxostheoremenv}[$\PaxosTheorem{2}$]
    If a proposal with value $v$ is chosen, then every higher-numbered proposal that is chosen has value $v$.
\end{paxostheoremenv}

Since numbers are totally ordered, condition $\PaxosTheorem{2}$ guarantees the crucial safety property that only a single value is chosen.
To be chosen, a proposal must be accepted by at least one acceptor.
So, we can satisfy $\PaxosTheorem{2}$ by satisfying

\begin{paxostheoremenv}[$\PaxosSubtheorem{2}{a}$]
    If a proposal with value $v$ is chosen, then every higher-numbered proposal accepted by any acceptor has value $v$.
\end{paxostheoremenv}

We still maintain $\PaxosTheorem{1}$ to ensure that some proposal is chosen.
Because communication is asynchronous, a proposal could be chosen with some particular acceptor $c$ never having received any proposal.
Suppose a new proposer "wakes up" and issues a higher-numbered proposal with a different value.
$\PaxosTheorem{1}$ requires $c$ to accept this proposal, violating $\PaxosSubtheorem{2}{a}$.
Maintaining both $\PaxosTheorem{1}$ and $\PaxosSubtheorem{2}{a}$ requires strengthening $\PaxosSubtheorem{2}{a}$ to:

\begin{paxostheoremenv}[$\PaxosSubtheorem{2}{b}$]
    If a proposal with value $v$ is chosen, then every higher-numbered proposal issued by any proposer has value $v$.
\end{paxostheoremenv}

Since a proposal must be issued by a proposer before it can be accepted by an acceptor, $\PaxosSubtheorem{2}{b}$ implies $\PaxosSubtheorem{2}{a}$, which in turn implies $\PaxosTheorem{2}$.
To discover how to satisfy $\PaxosSubtheorem{2}{b}$, let's consider how we would prove that it holds.
We would assume that some proposal with number $m$ and value $v$ is chosen and show that any proposal issued with number $n > m$ also has value $v$.
We would make the proof easier by using induction on $n$, so we can prove that proposal number $n$ has value $v$ under the additional assumption that every proposal issued with a number in $m\ldots(n - 1)$ has value $v$, where $i \ldots j$ denotes the set of numbers from $i$ through $j$.
For the proposal numbered $m$ to be chosen, there must be some set $C$ consisting of a majority of acceptors such that every acceptor in $C$ accepted it.
Combining this with the induction assumption, the hypothesis that $m$ is chosen implies:

\begin{paxostheoremenv}
    Every acceptor in $C$ has accepted a proposal with number in $m\ldots(n-1)$, and every proposal with number in $m\ldots(n-1)$ accepted by any acceptor has value $v$.
\end{paxostheoremenv}

Since any set $S$ consisting of a majority of acceptors contains at least one member of $C$, we can conclude that a proposal numbered $n$ has value $v$ by ensuring that the following invariant is maintained:

\begin{paxostheoremenv}[$\PaxosSubtheorem{2}{c}$]
    For any $v$ and $n$, if a proposal with value $v$ and number $n$ is issued, then there is a set $S$ consisting of a majority of acceptors such that either $(a)$ no acceptor in $S$ has accepted any proposal numbered less than n, or $(b)$ $v$ is the value of the highest-numbered proposal among all proposals numbered less than $n$ accepted by the acceptors in $S$.
\end{paxostheoremenv}

We can therefore satisfy $\PaxosSubtheorem{2}{b}$ by maintaining the invariance of $\PaxosSubtheorem{2}{c}$.
To maintain the invariance of $\PaxosSubtheorem{2}{c}$, a proposer that wants to issue a proposal numbered $n$ must learn the highest-numbered proposal with number less than $n$, if any, that has been or will be accepted by each acceptor in some majority of acceptors.
Learning about proposals already accepted is easy enough; predicting future acceptances is hard.
Instead of trying to predict the future, the proposer controls it by extracting a promise that there won't be any such acceptances.
In other words, the proposer requests that the acceptors not accept any more proposals numbered less than $n$.
This leads to the following algorithm for issuing proposals.

\begin{enumerate}
    \item A proposer chooses a new proposal number $n$ and sends a request to each member of some set of acceptors, asking it to respond with:
    \begin{enumerate}
        \item A promise never again to accept a proposal numbered less than $n$, and
        \item The proposal with the highest number less than $n$ that it has accepted, if any.
    \end{enumerate}
    I will call such a request a \emph{prepare} request with number $n$.
    \item If the proposer receives the requested responses from a majority of the acceptors, then it can issue a proposal with number $n$ and value $v$, where $v$ is the value of the highest-numbered proposal among the responses, or is any value selected by the proposer if the responders reported no proposals.
\end{enumerate}

A proposer issues a proposal by sending, to some set of acceptors, a request that the proposal be accepted.
(This need not be the same set of acceptors that responded to the initial requests.)
Let's call this an \emph{accept} request.
This describes a proposer's algorithm.
What about an acceptor?
It can receive two kinds of requests from proposers: \emph{prepare} requests and \emph{accept} requests.
An acceptor can ignore any request without compromising safety.
So, we need to say only when it is allowed to respond to a request.
It can always respond to a \emph{prepare} request.
It can respond to an \emph{accept} request, accepting the proposal, iff it has not promised not to.
In other words:

\begin{paxostheoremenv}[$\PaxosSubtheorem{1}{a}$]
    An acceptor can accept a proposal numbered $n$ iff it has not responded to a \emph{prepare} request having a number greater than $n$.
\end{paxostheoremenv}

Observe that $\PaxosSubtheorem{1}{a}$ subsumes $\PaxosTheorem{1}$.
We now have a complete algorithm for choosing a value that satisfies the required safety properties — assuming unique proposal numbers.
The final algorithm is obtained by making one small optimization.
Suppose an acceptor receives a prepare request numbered $n$, but it has already responded to a prepare request numbered greater than $n$, thereby promising not to accept any new proposal numbered $n$.
There is then no reason for the acceptor to respond to the new prepare request, since it will not accept the proposal numbered $n$ that the proposer wants to issue.
So we have the acceptor ignore such a prepare request.
We also have it ignore a prepare request for a proposal it has already accepted.
With this optimization, an acceptor needs to remember only the highest-numbered proposal that it has ever accepted and the number of the highest-numbered prepare request to which it has responded.
Because $\PaxosSubtheorem{2}{c}$ must be kept invariant regardless of failures, an acceptor must remember this information even if it fails and then restarts.
Note that the proposer can always abandon a proposal and forget all about it — as long as it never tries to issue another proposal with the same number.
Putting the actions of the proposer and acceptor together, we see that the algorithm operates in the following two phases.

\begin{phase}
    $(a)$ A proposer selects a proposal number $n$ and sends a \emph{prepare} request with number $n$ to a majority of acceptors.
    
    $(b)$ If an acceptor receives a \emph{prepare} request with number $n$ greater than that of any \emph{prepare} request to which it has already responded, then it responds to the request with a promise not to accept any more proposals numbered less than $n$ and with the highest-numbered proposal (if any) that it has accepted.
\end{phase}

\begin{phase}
    $(a)$ If the proposer receives a response to its \emph{prepare} requests (numbered $n$) from a majority of acceptors, then it sends an \emph{accept} request to each of those acceptors for a proposal numbered $n$ with a value $v$, where $v$ is the value of the highest-numbered proposal among the responses, or is any value if the responses reported no proposals.

    $(b)$ If an acceptor receives an \emph{accept} request for a proposal numbered $n$, it accepts the proposal unless it has already responded to a \emph{prepare} request having a number greater than $n$.
\end{phase}

A proposer can make multiple proposals, so long as it follows the algorithm for each one. 
It can abandon a proposal in the middle of the protocol at any time.
(Correctness is maintained, even though requests and/or responses for the proposal may arrive at their destinations long after the proposal was abandoned.)
It is probably a good idea to abandon a proposal if some proposer has begun trying to issue a higher-numbered one.
Therefore, if an acceptor ignores a \emph{prepare} or \emph{accept} request because it has already received a \emph{prepare} request with a higher number, then it should probably inform the proposer, who should then abandon its proposal.
This is a performance optimization that does not affect correctness.
