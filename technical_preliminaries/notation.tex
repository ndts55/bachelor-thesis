\section{Additional Notation}
\subsection{Prefixes and Branches} % TODO Find a better name for this subsection.
In \cite{PetersEtal21} the authors introduced notation for the construction of global types and processes.

Let $ {\left( \bigodot_{1 \leq i \leq n} \pi_i \right)}.\GT $ abbreviate the sequence $ \pi_1.\ldots.\pi_n.\GT $ to simplify the presentation, where $ \GT \in \globalTypes $ is a global type and $ \pi_1, \ldots, \pi_n $ are sequences of prefixes. More precisely, each $ \pi_i $ is of the form $ \pi_{i, 1}.\ldots.\pi_{i, m} $ and each $ \pi_{i, j} $ is a type prefix of the form $ \GComUS{\Role_1}{\Role_2}{\Label}{\Sort} $ or $ \GBranW{\Role}{\Role[R]}{\Label_1.\LT_1 \oplus \ldots \oplus \Label_n.\LT_n \oplus \LabelD} $, where the latter case represents a \weakR branching prefix with the branches $ \Label_1, \ldots, \Label_n, \LabelD $, the default branch $ \LabelD $, and where the next global type provides the missing specification for the default case.

Let $ \left( \bigodot_{1 \leq i \leq n} \pi_i \right).\PT $ abbreviate the sequence $ \pi_1.\ldots.\pi_n.\PT $, where $ \PT \in \processes $ is a process and $ \pi_1, \ldots, \pi_n $ are sequences of prefixes.

% TODO Do we need this notation for local types?
% Let $\DotForall{1\leq i \leq n}{\pi_i}.\Type{}{}$ abbreviate the sequence $\pi_i.\ldots.\pi_n.\Type{}{}$, where $\Type{}{}\in\mathcal{T}$ is a local type and $\pi_1,\ldots,\pi_n$ are sequences of prefixes.

\subsection{Function Calls as Prefixes}
% We extend ftmpst prefixes by function calls
We extend \FTMPST by adding function calls as prefixes.
These functions allow us to add side effects to our model.

% show syntax
Let $\Function{\GenericFunction}{\Domain}{\Image}$ be a function with domain $\Domain = \Domain_1\times\ldots\times\Domain_n$ and $\PT\in\processes$ a process.
Then $\GenericFunctionCall . \PT$ is also a process in $\processes$.

% show reduction rule
We introduce reduction rule \RSideEffect as $\GenericFunctionCall . \PT \longmapsto \PT$.
After the call to $\GenericFunction$, the process behaves like $\PT$.

% show typing rule that just ignores the function call
% % types of the arguments need to correspond to the types of the expressions used
Additionally, we define typing rule $\RSideEffect$.
Let $\SetOfFunctions$ be the set of functions $\Function{g}{\Domain}{\Image}$.
\begin{prooftree}
\AxiomC{$\GenericFunction\in\SetOfFunctions$}
\AxiomC{$\GammaSatisfiesType{\Elements}{\Domain}$}
\AxiomC{$\Gamma\vdash \PT\vartriangleright\Delta$}
\LeftLabel{\RSideEffect}
\TrinaryInfC{$\Gamma\vdash \GenericFunctionCall . \PT\vartriangleright\Delta$}
\end{prooftree}

% argument expressions have to satisfy corresponding type
% function has to take element of domain as argument
In \RSideEffect we require that $\GenericFunction$ is a function from $\Domain$ to $\Image$ and expression $\Elements$ is of the sort $\Domain$.
The latter implies that for $1 \leq i \leq n$ each sub-expression $\Element_i$ is of the sort $\Domain_i$ if all names $w$ in $\Element_i$ are replaced by arbitrary values of sort $S_w$ for $w:S_w \in \Gamma$.
The sorts of the arguments given correspond to the sorts of the arguments expected in $\GenericFunction$.
Applying \RSideEffect does not affect the global environment $\Gamma$ or the session environment $\Delta$.

The addition of function calls as prefixes has no significant impact on any proofs in \cite{PetersEtal21}.
