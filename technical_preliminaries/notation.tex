\section{Additional Notation}
In \cite{PetersEtal21} the authors introduced notation for the construction of global types and processes.

Let $ {\left( \bigodot_{1 \leq i \leq n} \pi_i \right)}.\GT $ abbreviate the sequence $ \pi_1.\ldots.\pi_n.\GT $ to simplify the presentation, where $ \GT \in \globalTypes $ is a global type and $ \pi_1, \ldots, \pi_n $ are sequences of prefixes. More precisely, each $ \pi_i $ is of the form $ \pi_{i, 1}.\ldots.\pi_{i, m} $ and each $ \pi_{i, j} $ is a type prefix of the form $ \GComUS{\Role_1}{\Role_2}{\Label}{\Sort} $ or $ \GBranW{\Role}{\Role[R]}{\Label_1.\LT_1 \oplus \ldots \oplus \Label_n.\LT_n \oplus \LabelD} $, where the latter case represents a \weakR branching prefix with the branches $ \Label_1, \ldots, \Label_n, \LabelD $, the default branch $ \LabelD $, and where the next global type provides the missing specification for the default case.

Let $ \left( \bigodot_{1 \leq i \leq n} \pi_i \right).\PT $ abbreviate the sequence $ \pi_1.\ldots.\pi_n.\PT $, where $ \PT \in \processes $ is a process and $ \pi_1, \ldots, \pi_n $ are sequences of prefixes.

% TODO Do we need this notation for local types?
% Let $\DotForall{1\leq i \leq n}{\pi_i}.\Type{}{}$ abbreviate the sequence $\pi_i.\ldots.\pi_n.\Type{}{}$, where $\Type{}{}\in\mathcal{T}$ is a local type and $\pi_1,\ldots,\pi_n$ are sequences of prefixes.
