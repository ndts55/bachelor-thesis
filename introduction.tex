\chapter{Introduction}

% coordinating processes?
% - processes that need to achieve one goal together
% - accessing shared data
% - ensure correct (distributed) computation
% - reliable distributed coordination
In distributed computing, it is often necessary for coordinating processes to reach consensus, i.e., agree on the value of some data that are needed during computation.
These processes agree on the same values to ensure correct computation, which necessitates a correct consensus algorithm.
Thus, proving the correctness of consensus algorithms is important.

Due to the presence of faulty processes consensus algorithms are designed to be fault-tolerant.
To achieve fault-tolerance these algorithms must satisfy the following properties: termination, integrity, and agreement\cite{dist_sys}.

Proving these properties can be complicated.
Dynamic analysis tools lead to big state-spaces so static analysis is preferable.
Multiparty Session Types are particularly interesting since session typing can ensure absence of communication errors and deadlocks, and protocol conformance\cite{mpstbd}.
However, to properly model unreliable communication between processes a fault-tolerant extension to Multiparty Session Types is necessary.

\citeauthor{ftmpst} developed such an extension.
