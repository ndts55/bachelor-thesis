\chapter{Introduction}
In distributed systems components on different computers coordinate and communicate via message passing to achieve a common goal.
Sometimes, to achieve this goal, the individual components need to reach consensus, \ie agree on the value of some data using a consensus algorithm.
For example in state machine replication or when deciding which database transactions should be committed in what order.
For such a distributed system to behave correctly the consensus algorithm needs to be correct.
Thus, analyzing consensus algorithms is important.

To achieve consensus, consensus algorithms must satisfy the following properties: validity, agreement, and termination \cite{CoulourisEtal01}.
Proving these properties can be complicated.
Model checking tools lead to big state-spaces so static analysis is preferable.
For static analysis Multiparty Session Types are particularly interesting because session typing can ensure protocol conformance and the absence of communication errors and deadlocks \cite{ScalasEtal18}.

Due to the presence of faulty processes and unreliable communication consensus algorithms are designed to be fault-tolerant.
% to better analyze we can't disregard fault-tolerance
% so we need a fault-tolerant extension to MPST
Modelling fault-tolerance is not possible using Multiparty Session Types, thus a fault-tolerant extension is necessary.
\citeauthor{PetersEtal21} developed such an extension called Fault-Tolerant Multiparty Session Types (\FTMPST).

In this work we will use \FTMPST to analyze the consensus algorithm Paxos, as described in \cite{Lamport01}.

% TODO talk about Paxos, overview of what it is and what it does
% TODO give outline of the following chapters

% \section{Task}
% \IMRADlabel{methods}
% The task for the upcoming work can be split into two subtasks.
% First, fault-tolerant Multiparty Session Types\cite{PetersEtal21} are used for the analysis of Paxos\cite{Lamport01}.
% Second, the applicability of fault-tolerant Multiparty Session Types to such analyses is evaluated.

% For the first subtask a model of the Paxos algorithm will be constructed.
% This model will be manually verified using the reduction rules for (fault-tolerant) Multiparty Session Types\cite{HondaYoshidaCarbone08,PetersEtal21}.

% The second part will be an evaluation of the applicability of fault-tolerant Multiparty Session Types to such analyses.
% Both the model and its verification from the previous part are evaluated to determine how well fault-tolerant Multiparty Session Types are suited to real-world analyses.

% \section{Possible difficulties}
% \IMRADlabel{results}
