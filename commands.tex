\newcommand{\TODO}[1]{\textcolor{red}{#1}}
\DeclareUnicodeCharacter{2212}{-}

% \newtheorem{condition}[corollary]{Condition}

%%%%%%%%%%%%%%%%%%%%%%%%%%%%%%%%%
%  abbreviations and key words  %
%%%%%%%%%%%%%%%%%%%%%%%%%%%%%%%%%

% makros for text
\newcommand{\ie}{i.e.,\xspace}
\newcommand{\eg}{e.g.\xspace}
\newcommand{\wrt}{w.r.t.\ }
\newcommand{\cf}{cf.\ }
\newcommand{\MPST}{MPST\xspace}
\newcommand{\piCal}{$ \pi $-calculus\xspace}
\newcommand{\PiCal}{$ \pi $-Calculus\xspace}
\newcommand{\strongR}{strong\-ly re\-li\-ab\-le\xspace}
\newcommand{\StrongR}{Strong\-ly Re\-li\-ab\-le\xspace}
\newcommand{\weakR}{weak\-ly re\-li\-ab\-le\xspace}
\newcommand{\WeakR}{Weak\-ly Re\-li\-ab\-le\xspace}
\newcommand{\unrel}{un\-re\-li\-ab\-le\xspace}
\newcommand{\Unrel}{Un\-re\-li\-ab\-le\xspace}

% makros for formulas
\newcommand{\iR}{\operatorname{r}}
\newcommand{\iU}{\operatorname{u}}
\newcommand{\iW}{\operatorname{w}}
\newcommand{\Set}[2][]{\left\lbrace #1 #2 #1 \right\rbrace}
\newcommand{\nat}{\mathbb{N}}
\newcommand{\deff}{\; \triangleq \;}
\newcommand{\deffTerms}{\; \mathop{::=} \;}
\newcommand{\sepTerms}{\hspace{0.75em} | \hspace{0.75em}}
\newcommand{\logdot}{. \;}
\newcommand{\Subst}[2]{{\Set{ \nicefrac{#1}{#2} }}}
\newcommand{\Length}[1]{\left| #1 \right|}
\newcommand{\indexSet}[1][I]{\mathrm{#1}}

%%%%%%%%%%%%%%%%%%%%%%%%%%%%%%
%  multiparty session types  %
%%%%%%%%%%%%%%%%%%%%%%%%%%%%%%

% names, variables, channels, ...
\newcommand{\default}{\operatorname{d}}
\newcommand{\roles}{\mathcal{R}}
\newcommand{\Role}[1][r]{\mathsf{#1}}
\newcommand{\Roles}[1]{\operatorname{R}\!\left( #1 \right)}
\newcommand{\Actors}[1]{\operatorname{A}\!\left( #1 \right)}
\newcommand{\labels}{\mathcal{L}}
\newcommand{\Label}[1][l]{\mathit{#1}}
\newcommand{\LabelD}{\Label_{\operatorname{d}}}
\newcommand{\LabelB}{\Label_{\operatorname{b}}}
\newcommand{\names}{\mathcal{N}}
\newcommand{\FreeNames}[1]{\operatorname{FN}\!\left( #1 \right)}
\newcommand{\BoundNames}[1]{\operatorname{BN}\!\left( #1 \right)}
\newcommand{\Args}[1][x]{\mathit{#1}}
\newcommand{\Chan}[1][s]{\mathit{#1}}
\newcommand{\typeVars}{\mathcal{V}_{\operatorname{T}}}
\newcommand{\TypeV}[1][t]{\mathit{#1}}
\newcommand{\procVars}{\mathcal{V}_{\operatorname{P}}}
\newcommand{\ProcV}[1][X]{\mathit{#1}}
\newcommand{\atomicTypes}{\mathcal{A}}
\newcommand{\AtomicType}[1]{\mathrm{#1}}
\newcommand{\sorts}{\mathcal{S}}
\newcommand{\Sort}[1][S]{\mathrm{#1}}
\newcommand{\expressions}{\mathcal{E}}
\newcommand{\Expr}[1][e]{#1}
\newcommand{\bool}{\mathbb{B}}
\newcommand{\true}{\mathtt{t}}
\newcommand{\false}{\mathtt{f}}

% global types
\newcommand{\globalTypes}{\mathcal{G}}
\newcommand{\GT}[1][G]{#1}
\newcommand{\GTD}{\GT_{\operatorname{d}}}
\newcommand{\GTB}{\GT_{\operatorname{b}}}
\newcommand{\GComR}[4]{#1 \to_{\iR} #2{:}{\left< #3 \right>}.#4}
\newcommand{\GComU}[5]{#1 \to_{\iU} #2{:}#3{\left< #4 \right>}.#5}
\newcommand{\GComUS}[4]{#1 \to_{\iU} #2{:}#3{\left< #4 \right>}}
\newcommand{\GBranR}[3]{#1 \to_{\iR} #2{:}{#3}}
\newcommand{\GBranW}[3]{#1 \to_{\iW} #2{:}{#3}}
\newcommand{\GPar}[2]{#1 \; || \; #2}
\newcommand{\GRep}[2]{{\left( \mu #1 \right)} #2}
\newcommand{\GEnd}{\mathtt{end}}
\newcommand{\GDel}[6]{#1 \to #2{:}{\left< \Typed{\AT{#3}{#4}}{#5} \right>}.#6}

% local types
\newcommand{\localTypes}{\mathcal{T}}
\newcommand{\LT}[1][T]{#1}
\newcommand{\LTD}{\LT_{\operatorname{d}}}
\newcommand{\LTB}{\LT_{\operatorname{b}}}
\newcommand{\LSendR}[3]{{\left[ #1 \right]}\mathsf{!}_{\iR}{\left< #2 \right>}.#3}
\newcommand{\LGetR}[3]{{\left[ #1 \right]}\mathsf{?}_{\iR}{\left< #2 \right>}.#3}
\newcommand{\LSendU}[4]{{\left[ #1 \right]}\mathsf{!}_{\iU}#2{\left< #3 \right>}.#4}
\newcommand{\LGetU}[4]{{\left[ #1 \right]}\mathsf{?}_{\iU}#2{\left< #3 \right>}.#4}
\newcommand{\LSelR}[2]{{\left[ #1 \right]}\mathsf{!}_{\iR}{#2}}
\newcommand{\LBranR}[2]{{\left[ #1 \right]}\mathsf{?}_{\iR}{#2}}
\newcommand{\LSelW}[2]{{\left[ #1 \right]}\mathsf{!}_{\iW}{#2}}
\newcommand{\LBranW}[2]{{\left[ #1 \right]}\mathsf{?}_{\iW}{#2}}
\newcommand{\LRep}[2]{{\left( \mu #1 \right)} #2}
\newcommand{\LEnd}{\mathtt{end}}
\newcommand{\LDelA}[5]{{\left[ #1 \right]}\mathsf{!}{\left< \Typed{\AT{#2}{#3}}{#4} \right>}.#5}
\newcommand{\LDelB}[5]{{\left[ #1 \right]}\mathsf{?}{\left< \Typed{\AT{#2}{#3}}{#4} \right>}.#5}

% session calculus
\newcommand{\processes}{\mathcal{P}}
\newcommand{\PT}[1][P]{\ensuremath{\mathit{#1}}}
\newcommand{\PTD}[1][P]{\PT[#1]_{\operatorname{d}}}
\newcommand{\PTB}[1][P]{\PT[#1]_{\operatorname{b}}}
\newcommand{\PReq}[4]{\overline{#1}{\left[ #2 \right]}{\left( #3 \right)}.#4}
\newcommand{\PAcc}[4]{#1{\left[ #2 \right]}{\left( #3 \right)}.#4}
\newcommand{\PSendR}[5]{#1{\left[ #2, #3 \right]}\mathsf{!}_{\iR}{\left< #4 \right>}.#5}
\newcommand{\PGetR}[5]{#1{\left[ #2, #3 \right]}\mathsf{?}_{\iR}{\left( #4 \right)}.#5}
\newcommand{\PSendU}[6]{#1{\left[ #2, #3 \right]}\mathsf{!}_{\iU}#4{\left< #5 \right>}.#6}
\newcommand{\PSendUS}[5]{#1{\left[ #2, #3 \right]}\mathsf{!}_{\iU}#4{\left< #5 \right>}}
\newcommand{\PGetU}[7]{#1{\left[ #2, #3 \right]}\mathsf{?}_{\iU}#4{\left< #5 \right>}{\left( #6 \right)}.#7}
\newcommand{\PGetUS}[6]{#1{\left[ #2, #3 \right]}\mathsf{?}_{\iU}#4{\left< #5 \right>}{\left( #6 \right)}}
\newcommand{\PSelR}[5]{#1{\left[ #2, #3 \right]}\mathsf{!}_{\iR}#4.#5}
\newcommand{\PBranR}[4]{#1{\left[ #2, #3 \right]}\mathsf{?}_{\iR}{#4}}
\newcommand{\PSelW}[5]{#1{\left[ #2, #3 \right]}\mathsf{!}_{\iW}#4.#5}
\newcommand{\PSelWS}[4]{#1{\left[ #2, #3 \right]}\mathsf{!}_{\iW}#4}
\newcommand{\PBranW}[4]{#1{\left[ #2, #3 \right]}\mathsf{?}_{\iW}{#4}}
\newcommand{\myif}{\mathtt{if}}
\newcommand{\mythen}{\mathtt{then}}
\newcommand{\myelse}{\mathtt{else}}
\newcommand{\PITE}[3]{\myif \; #1 \; \mythen \; #2 \; \myelse \; #3}
\newcommand{\PPar}[2]{#1 \mid #2}
\newcommand{\PDelA}[5]{#1{\left[ #2, #3 \right]}\mathsf{!}{\left<\!{\left< #4 \right>}\!\right>}.#5}
\newcommand{\PDelB}[5]{#1{\left[ #2, #3 \right]}\mathsf{?}{\left(\!{\left( #4 \right)}\!\right)}.#5}
\newcommand{\PRes}[2]{{\left( \nu #1 \right)} #2}
\newcommand{\PRep}[2]{{\left( \mu #1 \right)} #2}
\newcommand{\PEnd}{\mathbf{0}}
\newcommand{\PCrash}{\bot}
\newcommand{\Queue}[1][M]{\mathrm{#1}}
\newcommand{\MQ}[4]{#1_{#2 \to #3}{:}#4}
\newcommand{\MQS}[3]{#1_{#2 \to #3}}
\newcommand{\emptyList}{[\,]}

% semantics of processes
\newcommand{\step}{\longmapsto}
\newcommand{\steps}{\longmapsto^*}
\newcommand{\stepsP}{\longmapsto^+}
\newcommand{\nStep}{\; \not\!\!\longmapsto}
\newcommand{\compL}[1][\;]{#1\dot{=}#1}
\newcommand{\nCompL}[1][\;]{#1\not\!\!\dot{=}#1}
\newcommand{\Eval}[1]{{\operatorname{eval}}{\left( #1 \right)}}
\newcommand{\fpUGet}{\mathtt{FP}_{\mathtt{uget}}}
\newcommand{\fpUSkip}{\mathtt{FP}_{\mathtt{uskip}}}
\newcommand{\fpML}{\mathtt{FP}_{\mathtt{ml}}}
\newcommand{\fpWSkip}{\mathtt{FP}_{\mathtt{wskip}}}
\newcommand{\fpCrash}{\mathtt{FP}_{\mathtt{crash}}}

% messages
\newcommand{\messages}{\mathcal{M}}
\newcommand{\messageTypes}{\mathcal{MT}}
\newcommand{\MT}[1][MT]{\mathrm{#1}}
\newcommand{\MessR}[1]{{\left< #1 \right>^{\iR}}}
\newcommand{\MessU}[2]{#1{\left< #2 \right>^{\iU}}}
\newcommand{\MessBR}[1]{#1^{\iR}}
\newcommand{\MessBW}[1]{#1^{\iW}}

% well-typedness
\newcommand{\AT}[2]{#1{\left[ #2 \right]}}
\newcommand{\Typed}[2]{#1{:}#2}
\newcommand{\compS}{\cdot}
\newcommand{\Proj}[2]{{#1}{\restriction_{#2}}}
\newcommand{\Unreliable}[1]{{\operatorname{nsr}}{\left( #1 \right)}}

%%%%%%%%%%%%%%
%  Examples  %
%%%%%%%%%%%%%%

% example to explain the syntax
\newcommand{\GDice}{\GT_{\text{dice}, \iR}}
\newcommand{\GDiceUC}{\GT_{\text{dice}, \iU}}
\newcommand{\GDiceW}{\GT_{\text{dice}}}
\newcommand{\LDice}[1]{\LT_{\Role[#1]:\text{dice}, \iR}}
\newcommand{\LDiceW}[1]{\LT_{\Role[#1]:\text{dice}}}
\newcommand{\PDice}{\PT_{\text{dice}}}
\newcommand{\PDiceD}{\PT_{\Role[3]}}
\newcommand{\PDiceP}{\PT_{\Role[i]}}
\newcommand{\PDiceQ}{\PT_{\Role[2]}}
\newcommand{\PDiceR}{\PT_{\Role[1]}}
\newcommand{\PDiceUC}{\PT_{\text{dice}, \iU}}
\newcommand{\PDiceDUC}{\PT_{\Role[3], \iU}}
\newcommand{\PDicePUC}{\PT_{\Role[i], \iU}}
\newcommand{\PDiceQUC}{\PT_{\Role[2], \iU}}

% large example
\newcommand{\GTLE}[2][G]{\GT[#1]_{\operatorname{sub}}\!\left( #2 \right)}
\newcommand{\GTLEC}[2][G]{\GT[#1]_{\operatorname{ct}}\!\left( #2 \right)}
\newcommand{\SortBel}{\ensuremath{\Sort_{\mathrm{belief}}}}
\newcommand{\SortAck}{\ensuremath{\Sort_{\mathrm{ack}}}}
\newcommand{\RCSys}{\PT[Sys]\!\left( \Role[n], \InitKnowledge \right)}
%\newcommand{\RCP}[3][P]{\PT[#1]_{\operatorname{p_{1}}}\!\left( #2, #3 \right)}
\newcommand{\RCP}[2][i]{\PT[P]\!\left( \Role[#1], \Role[n], #2  \right)}
\newcommand{\RCPe}[2][i]{\PT[P]\!\left( #1, \Role[3], #2  \right)}
\newcommand{\RCPi}{\PT[P]_{\operatorname{1}}\!\left( \Role[i], \Role[n], r, \Knowledge \right)}
\newcommand{\RCPio}[2][C]{\PT[P]_{\operatorname{1}}^{\operatorname{#1}}\!\left( \Role[#2], \Role[n], \Role[c], \Knowledge \right)}
\newcommand{\RCPioe}[4][C]{\PT[P]_{\operatorname{1}}^{\operatorname{#1}}\!\left( #2, \Role[3], #3, #4 \right)}
\newcommand{\RCPiio}[3][C]{\PT[P]_{\operatorname{2}}^{\operatorname{#1}}\!\left( \Role[#2], \Role[n], \Role[c], #3 \right)}
\newcommand{\RCPiioe}[4][C]{\PT[P]_{\operatorname{2}}^{\operatorname{#1}}\!\left( #2, \Role[3], #3, #4 \right)}
\newcommand{\RCPiiio}[2][C]{\PT[P]_{\operatorname{3}}^{\operatorname{#1}}\!\left( \Role[#2], \Role[n], \Role[c], \Knowledge \right)}
\newcommand{\RCPiiioe}[4][C]{\PT[P]_{\operatorname{3}}^{\operatorname{#1}}\!\left( #2, \Role[3], #3, #4 \right)}
\newcommand{\RCPiiin}[4][NC]{\PT[P]_{\operatorname{3}}^{\operatorname{#1}}\!\left( \Role[#2], \Role[n], \Role[c], #3, #4 \right)}
\newcommand{\RCPiiine}[5][NC]{\PT[P]_{\operatorname{3}}^{\operatorname{#1}}\!\left( #2, \Role[3], #3, #4, #5 \right)}
\newcommand{\RCPiv}{\PT[P]_{\operatorname{4}}\!\left( \Role[i], \Role[n], r, \Knowledge \right)}
\newcommand{\RCPive}[3]{\PT[P]_{\operatorname{4}}\!\left( #1, \Role[3], #2, #3 \right)}
\newcommand{\RCPivo}[2][C]{\PT[P]_{\operatorname{4}}^{\operatorname{#1}}\!\left( \Role[#2], \Role[n], \Role[c], \Knowledge \right)}
\newcommand{\RCPivoe}[4][C]{\PT[P]_{\operatorname{4}}^{\operatorname{#1}}\!\left( #2, \Role[3], #3, #4 \right)}
\newcommand{\InitKnowledge}[1][]{\ensuremath{\vec{V^{0}_{\Role[#1]}}}} % todo, sieht scheiße aus
\newcommand{\Knowledge}{\ensuremath{\vec{V}}} %todo
\newcommand{\KnowledgeE}[2][]{\ensuremath{\vec{V_{#2}^{#1}}}}
\newcommand{\FuncBest}{\ensuremath{\operatorname{best}}}
\newcommand{\FuncReceived}{\ensuremath{\operatorname{received}}}
\newcommand{\FuncCountAck}{\ensuremath{\operatorname{ack}}}
\newcommand{\FuncUpdate}{\ensuremath{\operatorname{update}}}
\newcommand{\ceilTwo}[1]{\ensuremath{\bigg\lceil#1\bigg\rceil}}
\newcommand{\FDS}{\ensuremath{\Diamond \mathcal{S}}\xspace}
\newcommand{\PTMQ}[1][P]{\ensuremath{\PT[#1]_{\mathit{MQ}}}}
\newcommand{\ATE}[2]{\AT[]{\Chan[#1]\!}{\Role[#2]}}
\newcommand{\ImplMajor}{\ensuremath{\big\lceil\frac{\Role[n]-1}{2}\big\rceil}\xspace}

\newtheoremstyle{paxos}{}{}{\itshape}{}{\bfseries}{}{.5em}{\thmnote{#3}}
\theoremstyle{paxos}
\newtheorem*{paxostheoremenv}{}
\newtheorem*{reductionrule}{}

\theoremstyle{definition}
\newtheorem{phase}{Phase}

\newcommand{\PaxosTheoremConstructor}[2]{\operatorname{P#1^{#2}}}
\newcommand{\PaxosTheorem}[1]{\PaxosTheoremConstructor{#1}{}}
\newcommand{\PaxosSubtheorem}[2]{\PaxosTheoremConstructor{#1}{#2}}

% General typesetting stuff
\newcommand{\Curly}[1]{\left\{#1\right\}}
\newcommand{\Paren}[1]{\left(#1\right)}
\newcommand{\comment}[1]{\;\textcolor{teal}{#1}}
\def\IndentationSize{0.5em}
\newcommand{\Indent}[1]{\hspace*{\IndentationSize * #1}}

% Technical Preliminaries
\newcommand{\FTMPST}[0]{FTMPST\xspace}

% Sorts
\newcommand{\Bool}[0]{\operatorname{Bool}}
\newcommand{\True}[0]{\operatorname{true}}
\newcommand{\False}[0]{\operatorname{false}}
\newcommand{\Maybe}[1]{\operatorname{Maybe}\; #1}
\newcommand{\Just}[1]{\operatorname{Just}\; #1}
\newcommand{\Nothing}[0]{\operatorname{Nothing}}
\newcommand{\Or}[0]{\; | \;}
\newcommand{\Promise}[1]{\operatorname{Promise}\; #1}
\newcommand{\Proposal}[1]{\operatorname{Proposal}\; #1}
\newcommand{\ProposalC}[2]{\operatorname{Proposal}\; #1\; #2}
\newcommand{\Nack}[1]{\operatorname{Nack}\; #1}
\newcommand{\Value}[0]{\operatorname{Value}}

% MPST
%% General
\newcommand{\DotForall}[2]{\textstyle\left(\bigodot_{#1}\;#2\right)}
\newcommand{\Mu}[1]{\left(\mu #1\right)}
\newcommand{\AQ}[0]{A_Q}
\newcommand{\End}[0]{0}

%% Phases
\newcommand{\POneA}[0]{1a}
\newcommand{\POneB}[0]{1b}
\newcommand{\PTwoA}[0]{2a}
\newcommand{\PTwoB}[0]{2b}
\newcommand{\LOneA}[0]{l1a}
\newcommand{\LOneB}[0]{l1b}
\newcommand{\LTwoA}[0]{l2a}
\newcommand{\LTwoB}[0]{l2b}

%% Global Type
\newcommand{\ProposerIndex}[0]{p}
\newcommand{\AcceptorIndex}[0]{a}
\newcommand{\OuterSharedPoint}[0]{o}
\newcommand{\SharedPoint}[1]{b_{#1}}
\newcommand{\OuterGlobalType}[0]{\operatorname{G}}
\newcommand{\GlobalType}[0]{\operatorname{G_{\ProposerIndex, A_Q}}}
\newcommand{\SendUnreliableG}[4]{#1 \to_u #2 : #3 \left\langle #4 \right\rangle}
\newcommand{\SendWeaklyG}[3]{\left(#1 \to_w #2 : #3\right)}

%% Local Type
\newcommand{\SendUnreliableL}[3]{\left[#1\right]!_u #2 \left\langle #3 \right\rangle}
\newcommand{\ReceiveUnreliableL}[3]{\left[#1\right]?_u #2 \left\langle #3 \right\rangle}
\newcommand{\SendWeaklyL}[2]{\left(\left[#1\right]!_w #2\right)}
\newcommand{\ReceiveWeaklyL}[2]{\left(\left[#1\right]?_w #2\right)}
\newcommand{\OuterGlobalTypeProjection}[1]{\OuterGlobalType \upharpoonright_{#1}}
\newcommand{\GlobalTypeProjection}[1]{\GlobalType \upharpoonright_{#1}}

%% Process
\newcommand{\SendUnreliableP}[5]{#1\left[#2, #3\right]!_u #4 \left\langle #5 \right\rangle}
\newcommand{\ReceiveUnreliableP}[6]{#1\left[#2, #3\right]?_u #4 \left\langle #5 \right\rangle \left(#6\right)}
\newcommand{\SendWeaklyP}[5]{#1\left[#2, #3\right]!_w #4.#5}
\newcommand{\ReceiveWeaklyP}[4]{#1\left[#2, #3\right]?_w #4}

% Failure Patterns
\newcommand{\FPuget}[0]{\mathtt{FP_{uget}}}
\newcommand{\FPuskip}[0]{\mathtt{FP_{uskip}}}
\newcommand{\FPml}[0]{\mathtt{FP_{ml}}}
\newcommand{\FPwskip}[0]{\mathtt{FP_{wskip}}}
\newcommand{\FPcrash}[0]{\mathtt{FP_{crash}}}
\newcommand{\FailureDetectorClass}[0]{\Diamond\mathscr{S}}

% Paxos Process
\newcommand{\AcceptorCount}[0]{c_A}
\newcommand{\ProposerCount}[0]{c_P}
\newcommand{\Accept}[0]{\mathnormal{Accept}}
\newcommand{\Restart}[0]{\mathnormal{Restart}}
\newcommand{\Abort}[0]{\mathnormal{Abort}}
\newcommand{\Sys}[2]{\operatorname{Sys}\left(#1, #2\right)}
\newcommand{\Process}[2]{\operatorname{P^{#1}_{#2}}}
\newcommand{\Pa}[1]{\Process{A}{#1}}
\newcommand{\Pp}[1]{\Process{P}{#1}}
\newcommand{\PaOne}[0]{\Pa{1}}
\newcommand{\PaTwo}[0]{\Pa{2}}
\newcommand{\PaAccept}[0]{\Pa{acc}}
\newcommand{\PpInitName}[0]{\Pp{init}}
\newcommand{\PpInit}[5]{\PpInitName\left(#1, #2, #3, #4, #5\right)}
\newcommand{\PpInitShort}[0]{\PpInitName\left(\dots\right)}
\newcommand{\PaInitName}[0]{\Pa{init}}
\newcommand{\PaInit}[6]{\PaInitName\left(#1, #2, #3, #4, #5, #6\right)}
\newcommand{\PaInitShort}[0]{\PaInitName\left(\dots\right)}
\newcommand{\SessionRequest}[3]{\overline{#1}\left[#2\right]\left(#3\right)}
\newcommand{\SessionAccept}[3]{#1\left[#2\right]\left(#3\right)}
\newcommand{\ParallelFor}[1]{\Pi_{#1}\;}
\newcommand{\Acceptor}[1]{\ensuremath{\operatorname{A_{#1}}}}
\newcommand{\Proposer}[1]{\ensuremath{\operatorname{P_{#1}}}}

% Functions
\newcommand{\proposalNumberName}[0]{\operatorname{proposalNumber}}
\newcommand{\proposalNumber}[2]{\proposalNumberName_#2\left( #1\right)}

\newcommand{\promiseValueName}[0]{\operatorname{promiseValue}}
\newcommand{\promiseValue}[1]{\promiseValueName\left(#1\right)}

\newcommand{\anyNackName}[0]{\operatorname{anyNack}}
\newcommand{\anyNack}[1]{\anyNackName\left(#1\right)}

\newcommand{\promiseCountName}[0]{\operatorname{promiseCount}}
\newcommand{\promiseCount}[1]{\promiseCountName\left(#1\right)}

\newcommand{\greaterThanName}[0]{\operatorname{gt}}
\newcommand{\greaterThan}[2]{\greaterThanName\left(#1, #2\right)}

\newcommand{\greaterEqualName}[0]{\operatorname{ge}}
\newcommand{\greaterEqual}[2]{\greaterEqualName\left(#1, #2\right)}

\newcommand{\nFromProposalName}[0]{\operatorname{nFromProposal}}
\newcommand{\nFromProposal}[1]{\nFromProposalName\left(#1 \right)}

\newcommand{\genAqName}[0]{\operatorname{gen\AQ}}
\newcommand{\genAq}[3]{\genAqName\left(#1, #2, #3\right)}

\newcommand{\update}[2]{\operatorname{update} \left(#1, #2\right)}

% Math and Programming
\newcommand{\ceil}[1]{\Big\lceil #1 \Big\rceil}
\newcommand{\VectorV}[0]{\overrightarrow{V}}
\newcommand{\If}[1]{\operatorname{if}\xspace #1}
\newcommand{\Then}[1]{\operatorname{then}\xspace #1}
\newcommand{\Else}[1]{\operatorname{else}\xspace #1}
\newcommand{\tOr}[0]{\text{ or }}
\newcommand{\mapstostar}[0]{\longmapsto^*}
\newcommand{\Wildcard}[0]{\mathtt{\_}}
\newcommand{\EmptyList}[0]{\mathtt{[]}}
\newcommand{\List}[1]{\mathtt{list\ of\ } #1}
\newcommand{\ListPattern}[2]{\Paren{#1 \mathtt{\#} #2}}
\newcommand{\SetOfProposers}[0]{\mathbb{P}}
\newcommand{\SetOfPermanentlySuspectedProcesses}[0]{\mathbb{F}}
\newcommand{\Function}[3]{#1 : #2 \to #3}
\newcommand{\GenericFunction}[0]{f}
\newcommand{\Domain}[0]{X}
\newcommand{\Image}[0]{\emptyset}
\newcommand{\Element}[0]{x}
\newcommand{\Elements}[0]{\left(\Element_1,\ldots,\Element_n\right)}
\newcommand{\GenericFunctionCall}[0]{\GenericFunction\Elements}
\newcommand{\SetOfFunctions}[0]{F_{\Domain, \Image}}

% Example run
\newcommand{\Nu}[1]{\left(\nu #1\right)}
\newcommand{\OuterSessionQueues}[0]{\ParallelFor{1 \le k,l \le 2, k \neq l} t_{k\to l}:[]}
\newcommand{\InnerSessionQueues}[1]{\ParallelFor{1 \le k,l \le 4, k \neq l} #1_{k\to l}:[]}
\newcommand{\NuChannels}[0]{\Nu{t}\Nu{s}\Nu{r}}
\newcommand{\BeginPaCont}[0]{\Accept \dots \oplus \Restart .X \oplus \Abort . \End}
\newcommand{\ABC}[0]{\operatorname{abc}}

% Reduction / Typing Rules
\newcommand{\Rule}[1]{\ensuremath{\Paren{\operatorname{#1}}}\xspace}
\newcommand{\RInit}[0]{\Rule{Init}}
\newcommand{\RUsend}[0]{\Rule{USend}}
\newcommand{\RUget}[0]{\Rule{UGet}}
\newcommand{\RUskip}[0]{\Rule{USkip}}
\newcommand{\RWsel}[0]{\Rule{WSel}}
\newcommand{\RWbran}[0]{\Rule{WBran}}
%% Reduction Rules
\newcommand{\RWskip}[0]{\Rule{WSkip}}
\newcommand{\Rml}[0]{\Rule{ML}}
%% Typing Rules
\newcommand{\RReq}[0]{\Rule{Req}}
\newcommand{\RAcc}[0]{\Rule{Acc}}
\newcommand{\REnd}[0]{\Rule{End}}
\newcommand{\RIf}[0]{\Rule{If}}
\newcommand{\RPar}[0]{\Rule{Par}}
\newcommand{\RRec}[0]{\Rule{Rec}}
\newcommand{\RVar}[0]{\Rule{Var}}
\newcommand{\RSideEffect}[0]{\Rule{Func}}
\newcommand{\RIfT}[0]{\Rule{If-T}}
\newcommand{\RIfF}[0]{\Rule{If-F}}

% Type Check
\newcommand{\SEnvEntry}[3]{#1\left[#2\right]:#3}
\newcommand{\GEnvEntry}[2]{#1:#2}
\newcommand{\GammaX}[0]{\Gamma'}
\newcommand{\GammaXV}[0]{\GammaX'}
\newcommand{\GammaXN}[0]{\GammaX'}
\newcommand{\GammaXNP}[0]{\GammaXN'}
\newcommand{\EnvSatisfiesType}[3]{#1\Vdash\GEnvEntry{#2}{\mathbb{#3}}}
\newcommand{\GammaSatisfiesType}[2]{\EnvSatisfiesType{\Gamma}{#1}{#2}}

\newcommand{\ProofLabel}[1]{\Paren{#1}}
\newcommand{\SysProof}[1]{\ProofLabel{S_{#1}}}
\newcommand{\ProposerProof}[1]{\ProofLabel{P_{#1}}}
\newcommand{\AcceptorProof}[1]{\ProofLabel{A_{#1}}}

\newcommand{\ProposerProofTrue}[0]{\ProposerProof{t}}
\newcommand{\ProposerProofFalse}[0]{\ProposerProof{f}}
\newcommand{\AcceptorProofOne}[0]{\AcceptorProof{1}}
\newcommand{\AcceptorProofTwo}[0]{\AcceptorProof{2}}
\newcommand{\AcceptorProofTrue}[0]{\AcceptorProof{t}}
\newcommand{\AcceptorProofFalse}[0]{\AcceptorProof{f}}
\newcommand{\AcceptorProofAccept}[0]{\AcceptorProof{\Accept}}

\newcommand{\Type}[2]{\ensuremath{\operatorname{T^{#1}_{#2}}}}
\newcommand{\Tp}[1]{\Type{P}{#1}}
\newcommand{\Ta}[1]{\Type{A}{#1}}
\newcommand{\TpBranch}[0]{\Tp{branch}}
\newcommand{\TpAccept}[0]{\Tp{acc}}
\newcommand{\TaOneB}[0]{\Ta{1b}}
\newcommand{\TaBranch}[0]{\Ta{branch}}
\newcommand{\TaAccept}[0]{\Ta{acc}}
